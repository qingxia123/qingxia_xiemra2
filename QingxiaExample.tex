\documentclass{ximera}
%% handout
%% space
%% newpage
%% numbers
%% nooutcomes

\input{preamble.tex} %% we can turn off input when making a master document

\outcome{Understand a first example of the Ximera style.}
\outcome{Have a nice basic example to work from.}

\title{Introduction to Real Numbers}


\begin{document}
\begin{abstract}
This section covers the notes of real numbers. 1
\end{abstract}

{\LARGE \textbf{1.4 Laws of Exponents and Scientific Notation}}

\noindent \rule{7.35in}{2pt}

\vspace{.2in}

\noindent \begin{tabular}{|c|} \hline Definition\\ \hline \end{tabular} $~~$ For any real number $x$ and any positive integer $a$,

\begin{center} {\LARGE $\displaystyle x^a=$} \end{center}

$~$

$~$

$~~~~~~~$where $x$ is the \rule{1in}{.5pt} and $a$ is the \rule{2in}{.5pt}.

$~$

$~$

\noindent \underline{Exponent of 1}: For any real number $x$, $x^1=$

$~$

\noindent \underline{Exponent of 0}: For any non zero real number $x$, $x^0=$

$~$

\noindent Ex 1) Find the indicated value of the expression.

$~$

\textbf{a.}$~~~~$$\displaystyle -3y^1$ \hspace{.75in} \textbf{b.}$~~~~$$\displaystyle -(2x)^0$ \hspace{.75in} \textbf{c.}$~~~~$$\displaystyle 5x^0$ \hspace{.75in} \textbf{d.}$~~~~$$\displaystyle (-7a+b)^1$

\vspace{1in}

\begin{center} \begin{tabular}{|l|}
\hline 
\textbf{Product Rule of Exponents}\\
\hline
For any nonzero real number $x$ and for any positive integers $a$ and $b$\\
$~$\\
$~~~~~~~~~~~~~~~~~~~~~~~~~~~~~~~~~~~~~${\Large$x^a\cdot x^b=~~~~$}\\
$~$\\
\hline
\end{tabular} \end{center}

\noindent Ex 2) Express in terms of a base raised to a single power if possible.

$~$

\textbf{a.}$~~~~$$\displaystyle n^3 \cdot n^7$\hspace{1in} \textbf{b.}$~~~~$$\displaystyle (-4x)^3 \cdot (-4x)$\hspace{1in}
\textbf{c.}$~~~~$$\displaystyle (y+3)^5 \cdot (y+5)^0$\hspace{1in} \textbf{d.}$~~~~$$\displaystyle p^6 \cdot q^3$

\vspace{.5in}

\begin{center} \begin{tabular}{|l|}
\hline 
Quotient Rule of Exponents\\
\hline
For any nonzero real number $x$ and for any positive integers $a$ and $b$\\
$~$\\
$~~~~~~~~~~~~~~~~~~~~~~~~~~~~~~~~~~~~~~~~~${\Large$\displaystyle \frac{x^a}{x^b}=~~~~$}\\
$~$\\
\hline
\end{tabular} \end{center}

\noindent Ex 3) Rewrite each expression as a base to a power if possible.

$~$

\textbf{a.}$~~~~$$\displaystyle \frac{s^8}{s^2}$\hspace{2in} \textbf{b.}$~~~~$$\displaystyle \frac{(-3r)^{10}}{(-3r)^9}$

\vspace{.5in}

\hspace{1in}\textbf{c.}$~~~~$$\displaystyle \frac{(t-2)^6}{(t-2)^4}  $\hspace{2in} \textbf{d.}$~~~~$$\displaystyle \frac{a^4}{b}$


\pagebreak

\begin{center} \begin{tabular}{|l|}
\hline 
Negative Exponents\\
\hline
For any nonzero real number $x$ and for any integer $a$\\
$~$\\
$~~~~~~~~~~~~~~~~~~~~~~~~~~${\Large$\displaystyle x^{-a} = $}\\
$~$\\
\hline
\end{tabular} \end{center}


\noindent Ex 4) Rewrite with a positive exponent.

$~$

\textbf{a.}$~~~~$$\displaystyle a^{-9}$\hspace{1in} \textbf{b.}$~~~~$$\displaystyle (-4y)^{-1}$\hspace{1in}
\textbf{c.}$~~~~$$\displaystyle (-5+x)^{-3}  $\hspace{1in} \textbf{d.}$~~~~$$\displaystyle \frac{1}{p^{-4}}$

\vspace{1in}

\begin{center} \begin{tabular}{|l|}
\hline 
Reciprocal of $x^{-a}$\\
\hline
For any nonzero real number $x$ and for any positive integer $a$\\
$~$\\
$~~~~~~~~~~~~~~~~~~~~~~~~~~~~~~~~${\Large$\displaystyle \frac{1}{x^{-a}} = $}\\
$~$\\
\hline
\end{tabular} \end{center}

\noindent Ex 5) Rewrite as expressions as using only positive exponents.

$~$

\textbf{a.}$~~~~$$\displaystyle \frac{1}{y^{-3}}$\hspace{2in} \textbf{b.}$~~~~$$\displaystyle \frac{5}{x^{-2}}$

\vspace{.5in}

\hspace{1in}\textbf{c.}$~~~~$$\displaystyle \frac{m}{n^{-3}}  $\hspace{2in} \textbf{d.}$~~~~$$\displaystyle \frac{4r^3}{s^{-3}}$

\vspace{.5in}

\begin{center} \begin{tabular}{|l|}
\hline 
Power Rule of Exponents\\
\hline
For any nonzero real number $x$ and for any integers $a$ and $b$\\
$~$\\
$~~~~~~~~~~~~~~~~~~~~~~~~~~~~~~${\Large$\displaystyle \left(x^a\right)^b=~~~~$}\\
$~$\\
\hline
\end{tabular} \end{center}


\noindent Ex 6) Simplify.

$~$

\textbf{a.}$~~~~$$\displaystyle (y^7)^2$\hspace{1.3in} \textbf{b.}$~~~~$$\displaystyle -(n^{-1})^5$\hspace{1.3in}
\textbf{c.}$~~~~$$\displaystyle (p^{-3})^{-3} $ 

\pagebreak

\begin{center} \begin{tabular}{|l|}
\hline 
Raising a Product to a Power\\
\hline
For any nonzero real numbers $x$ and $y$ and any integer $a$\\
$~$\\
$~~~~~~~~~~~~~~~~~~~~~~~~~~~~${\Large$\displaystyle \left(xy\right)^a=~~~~$}\\
$~$\\
\hline
\end{tabular} \end{center}

\noindent Ex 7) Simplify.

$~$

\textbf{a.}$~~~~$$\displaystyle (-4x)^2$\hspace{1.75in} \textbf{b.}$~~~~$$\displaystyle (5a^9)^2$\hspace{1.75in}\textbf{c.}$~~~~$$\displaystyle (-q^7r^8)^2  $

\vspace{.75in}

\hspace{1in}\textbf{d.}$~~~~$$\displaystyle -6(a^5b^3)^3  $\hspace{2in} \textbf{e.}$~~~~$$\displaystyle (7x^{-4}y^{-1})^{-2}$

\vspace{.5in}

\begin{center} \begin{tabular}{|l|}
\hline 
Raising a Quotient to a Power\\
\hline
For any nonzero real numbers $x$ and $y$ and any integer $a$\\
$~$\\
$~~~~~~~~~~~~~~~~~~~~~~~~~~~~~~${\Large$\displaystyle \left(\frac{x}{y}\right)^a=~~~~$}\\
$~$\\
\hline
\end{tabular} \end{center}

\noindent Ex 8) Simplify.

$~$

\textbf{a.}$~~~~$$\displaystyle \left(\frac{y}{2}\right)^5 $\hspace{2in} \textbf{b.}$~~~~$$\displaystyle \left(\frac{u^4}{v^6}\right)^2$

\vspace{1in}

\hspace{1in}\textbf{c.}$~~~~$$\displaystyle \left(\frac{-5a^5}{2b^2}\right)^3  $\hspace{2in} \textbf{d.}$~~~~$$\displaystyle \left(\frac{v^6}{u^4}\right)^{-2}$

\vspace{1in}

\begin{center} \begin{tabular}{|l|}
\hline 
Raising a Quotient to a Negative Power\\
\hline
For any nonzero real numbers $x$ and $y$ and any integer $a$\\
$~$\\
$~~~~~~~~~~~~~~~~~~~~~~~~~~~~~~${\Large$\displaystyle \left(\frac{x}{y}\right)^{-a}=~~~~$}\\
$~$\\
\hline
\end{tabular} \end{center}


\noindent Ex 9) Simplify. $~~~~$$\displaystyle \left( \frac{3x^3}{2y^2}\right)^{-4}$

\pagebreak


\noindent \begin{tabular}{|l|}
\hline 
Definition\\
\hline \end{tabular} A number is in \rule{3in}{.5pt} if it is written in the form

{\Large $$a \times 10^n$$}

where $n$ is an integer and $a$ is greater than or equal to 1 but less than 10 ($1\leq a < 10$)

\vspace{.2in}

\noindent \underline{Note}: With scientific notation, positive powers represent \rule{1in}{.5pt} numbers while negative powers represent \rule{1in}{.5pt} numbers.

$~$

\noindent \underline{Note}: When converting a number from scientific notation to standard notation, move the decimal point to the \rule{1in}{.5pt} if the power of 10 is \textit{positive} and to the \rule{1in}{.5pt} if the power of 10 is \textit{negative}.

$~$

\noindent Ex 10) Express in standard notation.

$~$

\textbf{a.} $~~~~$$\displaystyle 5.193 \times 10^8$ \hspace{2in} \textbf{b.}$~~~~$$\displaystyle 4.82 \times 10^{-7}$

\vspace{.75in}

\noindent Ex 11) Write each number in scientific notation.

$~$

\textbf{a.} $~~~~$$\displaystyle 4,000,000,000,000$ \hspace{2in} \textbf{b.}$~~~~$$\displaystyle 0.000000000067$

\vspace{.75in}

\noindent Ex 12) Carry out the computation.  Express the result in scientific notation.

$~$

\textbf{a.} $~~~~$$\displaystyle (4 \times 10^4)(7 \times 10^3)$ \hspace{2in} \textbf{b.}$~~~~$$\displaystyle (1.32 \times 10^4) \div (6 \times 10 ^{-4})$




\end{document}
